\section*{Введение}
% добавляем введение в содержание
\addcontentsline{toc}{section}{Введение}
Методы машинного обучения значительно помогаеют в автоматизации многиъ процессов, с некоторого времени, алгоритмы машинного и глубинного обучения стали обладать достаточными возможностями и надежностью для применения и в сфере медицины \cite{med_survey}. \newline
Медицина является достаточно обширной сферой как науки так и поведневной деятельности, ее аналитическая часть часто взаимодействует с большим количесвтом наблюдений разного рода, как в масштабах популяций, так и одного пациента. Даже один единственный пациент может пройти такое количество анализов, что понадобится достаточно продолжительное время и целый набор специалистов для постановки диагноза. Медицинские обследования пораждают данные в разных формах, включая, но не ограничиваяь табличными данными на разного рода тестах, визуальной информацией в в виде снимков, временных рядов при анализе разного рода биоритмов. 
Но еще до такого рода информации, о пациенте получают более сложно поддающуюся анализу, но легкодоступную ифнормацию - из разговор при визите ко врачу.\newline
Из рассказа пациента о своем состоянии, можно сделать выводы о возможных заболеваниях, вычленить ключевые и формализованные симптомы, которые понадобятся для ведения истории и дальнейшего направления на анализы и ведения пациента в целом. Бывшая актуальной во все времена проблема распознавания такого рода обостряется с развитием телемедицины - то есть обращении при отсутствии очного посещения врача. Доступность пораждает больший спрос и задача формализации каждого случая становится еще острее. Более того, процесс может быть автоматизирован рекоммендательной интеллектуальной системой для дальнейшего направления, или предложения направиться на анализы, тогда и для нее понадобится формальный список аномалий в состоянии пациента.
\newline
Только недавние достижения искуственного интеллекта позволили эффективно и надежно работать с такими данными. Работа с текстами на "человеческом языке" называется "обработкой естественного языка" (далее - Natural Language Processing или сокращенно NLP). значительный прорыв в области обработки такого рода данных связан с развитием глубинного обучения и дальнейшим появлением рекуррентных сетей и архитектур на основе механизма внимания.
\newline
Также важно отметить проблему сокрытости зананий нейронной сети. Эти алгоритмы представляют Black Box функции, то есть неизвестен смысл их внутренних процессов. Это неприемлимо при применении в задачах реальной жизни, тем более в медецинской сфере. Поэтому требуется дополнительное исследование и интерпретация обученной модели. В случае имплементации в рабочий процесс, эксперты должны иметь возможность оценивать, как алгоритм работает с данными для валидации верности работы, для этого не достаточно обычных метрик.
\newline
Целью работы является интерпретация и визуализация внутренних состояний нейронной сети на основе BERT. Для этого в работе преследуются следующие задачи: 
\begin{enumerate}
\item  Формализация NLP-задачи для ML и обзор технологий для решения.
\item  Механизмы стоящие за RNN и Transformer, изучение основных механизмов и архитектур для актуальных нейронных сетей в сфере NLP.
\item  Осуществить файн-тюнинг на основе претренированной языковой модели BERT для решения задачи классификации на медицинских данных в виде естественного языка, получив таким образом модель для исследования.
\item  Выбор методики анализа алгоритмов глубинного обучения, интерпретация с ее помощью полученной модели.
\end{enumerate}